\documentclass{article}
\usepackage{graphicx} % Required for inserting images
\usepackage{amsmath}
\usepackage{amssymb}
\usepackage{algorithm}
\usepackage{algpseudocode} % Use this for pseudocode
\usepackage[margin=1in]{geometry}

\title{CSC263 A5}
\author{Nena Harsch and Natalia Tabja}
\date{October 2024}

\begin{document}

\maketitle

\section*{Question 1 (written by Nena,  verified by Natalia)}
\subsection*{Part (a)}
We first want to define that $\$1$ pays for one pairwise comparison. From the question, we know that the actual cost of add is 0 and reduce is $\alpha{n}$. We will assign a charge of $2\alpha$ to the add operation and $\$0$ to the reduce operation.

\subsection*{Part (b)}
We will define two state invariant (SI) as follows:
\begin{enumerate}
    \item Each element in the list stores $\$2\alpha$.
    \item The sum of the charges of ops $\geq$ The sum of costs
\end{enumerate}
Proof:
\begin{itemize}
    \item Base Case: The size of the list is 0\\
        \begin{enumerate}
            \item We know that the list contains 0 elements, therefore 1 holds as there are 0 charges stored.
            \item The sum of charges = the sum of costs = 0 since the list is empty, therefore 2 holds.
        \end{enumerate}
    \item Induction Step: Suppose that the SIs hold for $m-1$ operations [IH]. We want to prove that the SIs hold for the $m^{th}$ operation.
        \begin{itemize}
            \item The $m^{th}$ operation is add.
                \begin{enumerate}
                    \item For a list of $n$ elements, by the IH and SI1 we have $\$2\alpha{n}$ charges. When we add another element, by our definition in part a we will charge another $\$2\alpha$. Therefore our list will now have a total of $\$2\alpha(n+1)$ charges, and since we have $n+1$ elements, SI1 holds.
                    \item There was an additional charge of $2\alpha$ when we added an element and no additional costs. Therefore, \\
                    sum of charges of m ops = sum of charges of m-1 ops + $2\alpha$ $>$ sum of charges of m-1 ops \\
                    sum of costs of m ops = sum of costs of m-1 ops \\
                    and thus by this, the IH, and SI2, we can conclude \\
                    The sum of the charges of m ops $>$ sum of charges of m-1 ops $\geq$ The sum of costs of m -1 ops = The sum of costs of m ops \\
                    The sum of the charges of m ops $\geq$ The sum of costs of m ops\\
                    Thus, SI2 holds.
                 \end{enumerate}
            \item The $m^{th}$ operation is reduce
                \begin{enumerate}
                    \item For a list of n elements, we have $2\alpha{n}$ charges stored in the list and diminishing the list has an actual cost of $\alpha{n}$. Therefore, SI2 holds since the sum of the charges is greater than the sum of the costs.
                    \item Since we diminish the list in half, there are $\lfloor{n/2}\rfloor$ elements and $\alpha{n}$ charges left. $\alpha{n}$  = $2\alpha * n/2 \geq 2\alpha * \lfloor{n/2}\rfloor$ charges, therefore SI1 holds.
                \end{enumerate}
        \end{itemize}
        \item Therefore, we can conclude that our state invariants are correct.
\end{itemize}

\subsection*{Part (c)}
T(n) is the worst case of doing any sequence of n operations (over all possible sequences of n operations). If we were to have a sequence of n add operations, by the accounting method and our credit invariants, we would have the maximum charge of $2\alpha{n}$ in credit (by SI1). For a sequence of n operations (any combination of add or reduce), we can not get a credit higher than this since a reduce would have a charge of 0. We also know that the sum of charges $\geq$ sum of costs (by SI2), so the sum of costs of n operations is bounded above by $2\alpha{n}$. Therefore, we can say that T(n)/n $\leq$ $2\alpha{n}$/n = $2\alpha$, which is constant. Thus, the amortized cost of an operation is constant.

\newpage
\section*{Question 2 (written by Natalia,  verified by Nena)}

\subsection*{Part (a)}
Let \( n \in \mathbb{N} \) be the size of the array (i.e., the number of elements in \( S \)).

\begin{itemize}
    \item \textbf{Search(x)}: The worst-case time complexity of \texttt{Search(x)} is \( O(\log n) \) if \( x \) is not in \( S \). We can achieve this by using binary search, which is feasible because \( A \) is sorted.
    \item \textbf{Insert(x)}: The worst-case time complexity of \texttt{Insert(x)} is \( O(n) \). Finding the correct position for \( x \) can be achieved in \( \log(n) \) time with binary search, but shifting elements in the array takes \( O(n) \) time if the element needs to be inserted at the beginning.
    \item \textbf{Amortized Insertion Time}: The amortized insertion time, defined as the worst-case total time to execute a sequence of \( n \) \texttt{Insert} operations divided by \( n \), is calculated as follows.

    Since \( S \) starts off empty, the first insertion takes \( 1 \) step. The second insertion takes \( 2 \) steps because we shift one element and insert the new one. The third insertion takes \( 3 \) steps, and so on. Thus, the total number of steps can be calculated as:
    \[
    1 + 2 + 3 + \dots + n = \sum_{i=1}^n i = \frac{n^2 + n}{2}
    \]
    Therefore, the amortized cost is:
    \[
    \frac{1}{n} \left( \frac{n^2 + n}{2} \right) = \frac{n + 1}{2}
    \]
    which is \( O(n) \).
\end{itemize}

\subsection*{Part (b)}
For two examples with different sets \( S \):
\begin{itemize}
    \item \textbf{i)} \( S = \{4, 6, 2, 11, 7\} \)
    \[
    n = 5 = \langle 1,0,1 \rangle \Rightarrow \text{Two arrays: one of size } 2^1 = 1 \text{ and one of size } 2^2 = 4:
    \]
    \[
    A_0: \{4\}, \quad A_2: \{2, 6, 7, 11\}
    \]

    \item \textbf{ii)} \( S = \{16, 7, 2, 9, 0, 11, 5\} \)
    \[
    n = 7 = \langle 1,1,1 \rangle \Rightarrow \text{Three arrays of sizes 1, 2, and 4:}
    \]
    \[
    A_0: \{16\}, \quad A_1: \{2, 7\}, \quad A_2: \{0, 5, 9, 11\}
    \]
\end{itemize}

\subsection*{Part (c)}
\textbf{Search(x)}:

\begin{enumerate}
    \item Starting from the first array, check the head of each array in \( L \).
    \item If \( x \) is smaller than the head of the array \( A_i \), we know \( x \notin A_i \), as each array is sorted in ascending order. Otherwise, we perform binary search on \( A_i \).
    \item If \( A_{i+1} \) doesn’t exist, we return \texttt{False}.
\end{enumerate}

\textbf{Time Complexity:}

- We check at most \( k = \lceil \log_2(n) \rceil \) arrays and perform binary search in each.
- The total number of steps is:
  \[
  \sum_{i=0}^{k-1} i = \frac{(k-1)k}{2} = O(\log^2 n)
  \]
  
Thus, the complexity of \texttt{Search(x)} is \( O(\log^2 n) \).

\subsection*{Part (d)}
\textbf{Insert(x):}

\begin{enumerate}
    \item Start by creating an array of length \( 1 = 2^0 \), with \( x \) as its only element.
    \item \textbf{Merging:}
    \begin{itemize}
        \item If \( L \) has no \( A_0 \) (i.e., no array of length \( 2^0 = 1 \)), then the new array becomes \( L \)'s \( A_0 \).
        \item Otherwise, merge it with the existing \( A_0 \) to form \( A_1 \) (an array of length \( 2^1 = 2 \)), and continue similarly for higher orders. Each time two arrays of the same order are merged, they form an array of the next highest order.
    \end{itemize}
    \item Continue merging until reaching an order \( i \) such that \( A_i \notin L \). Add the newly created array as \( A_i \) in \( L \), thus updating \( L \) with this array.
\end{enumerate}

\textbf{Sorted Merging:}
\begin{itemize}
    \item Initialize two pointers: one at the beginning of each array being merged.
    \item Compare the elements at each pointer; the smallest gets added to a new array, and the pointer corresponding to that element is incremented to the next element in the array.
    \item Continue until all elements are added to the new array.
\end{itemize}

\subsubsection*{Time Complexity Analysis}
\begin{itemize}
    \item \textbf{Step 1} takes constant time as it simply creates a new array.
    \item \textbf{Steps 2 and 3} involve merging approximately \( \log_2(n) \) arrays in the worst case, as this is the maximum number of pre-existing arrays that may need merging.
    \item Each merge operation between two arrays of size \( 2^i \) requires \( O(2^i) \) comparisons since each element in both arrays must be moved to the new array.
\end{itemize}

Thus, the total time for a single insertion is:
\[
O\left( \sum_{i=0}^{\log_2(n)-1} 2^{i+1} \right) = 2 \cdot O\left( \sum_{i=0}^{\log_2(n)-1} 2^i \right)
\]

Since \( \sum_{i=0}^{\log_2(n)-1} 2^i \) is a geometric series, we can simplify:
\[
= 2 \cdot O\left( 2^{\log_2(n)} - 1 \right) = 2 \cdot O(n - 1) = O(n)
\]

Therefore, the worst-case time complexity for a single insertion is \( O(n) \).



\subsection*{Part (e)}
\textbf{Amortized Complexity of Insert(x)} \\
\textit{Accounting Method:}
\begin{itemize}
    \item We charge each insertion \( O(\log_2(n)) \) credits, since each insertion could potentially trigger merges at multiple levels up to \( \log_2(n) \). This way, we are preparing for possible costs across all levels.
    \item Over \( n \) insertions, this results in a total of \( O(n \log_2(n)) \) credits, ensuring that we accumulate sufficient funds to pay for merges at every level when they occur.
    \item For each level \( i \), arrays of size \( 2^i \) are merged at a cost of \( O(2^{i+1}) \) operations. Since merges at level \( i \) happen only once every \( 2^i \) insertions (when the current level reaches capacity and needs to merge upward), we distribute the cost of these merges across the necessary number of insertions.
    \item The \( O(\log_2(n)) \) charge per insertion ensures that each insertion contributes enough credits to cover merges across all levels. Specifically, by the time we need to merge at level \( i \), we have accumulated exactly the \( 2^{i+1} \) credits required to cover the merge cost at that level.
\end{itemize}

Thus, the total credits accumulated after \( n \) insertions is \( O(n \log_2(n)) \). Dividing by \( n \), we find that the amortized cost per insertion is \( O(\log_2(n)) \), as required. \\

\textit{Aggregate Analysis:}
\begin{itemize}
    \item When we insert $n$ elements, we’ll have gone through each level of merging up to $\log_2(n)$ times.
    \item Each level/order $i$ costs $O(2^{i+1})$, but it only occurs $O(\frac{n}{2^i})$ times across all $n$ insertions.
    \item This is because to create an array at level $i$, we need $2^i$ elements. Each time we get $2^i$ elements, a merge is triggered at level $i$ because we reach the capacity of that level. This will occur approximately $\frac{n}{2^i}$ times.
\end{itemize}

Thus, the total cost for all $n$ insertions can be approximated by:
\[
\sum_{i=0}^{\log_2(n) - 1} \left(\frac{n}{2^i} \cdot 2^{i+1}\right)
= \sum_{i=0}^{\log_2(n) - 1} (2 \cdot n)
= 2n \cdot \log_2(n) \sim O(n \log_2(n))
\]

Thus on average, each insertion takes:
\[
\frac{O(n \log_2(n))}{n} = O(\log_2(n))
\]

\subsection*{Part (f)}
\textbf{Delete(x)}

\begin{enumerate}
    \item \textbf{Locate \( x \)} using \texttt{Search(x)}:
    \begin{itemize}
        \item Perform \texttt{Search(x)} to find the array \( A_i \) containing \( x \).
        \item \texttt{Search(x)} has a time complexity of \( O(\log^2 n) \).
        \item If \texttt{Search(x)} returns \texttt{False}, \( x \) is not in \( L \), and the delete operation terminates.
    \end{itemize}

    \item \textbf{Remove \( x \) from the Array}:
    \begin{itemize}
        \item Once \( x \) is found, remove it from \( A_i \).
        \item After removal, the array size is now \( 2^i - 1 \), which is not a power of 2.
    \end{itemize}

    \item \textbf{Handle the Empty Slot}:
    \begin{itemize}
        \item The array no longer satisfies the required power-of-2 size, so we cannot leave an empty slot in place.
        \item To restore the structure, we proceed to split this array into smaller arrays of sizes that are powers of 2.
    \end{itemize}

    \item \textbf{Divide Elements into New Arrays}:
    \begin{itemize}
        \item Rather than splitting the existing array in place, we distribute its \( 2^i - 1 \) elements into newly created arrays of sizes that are powers of 2.
        \item For example, an array of size 7 would be divided into new arrays of sizes 4, 2, and 1, containing the elements of the original array.
        \item The runtime for this step is \( O(2^i) \), which is \( O(n) \) in the worst case when the largest array is involved.
    \end{itemize}

    \item \textbf{Handle Duplicate Sizes with Cascading Merges}:
    \begin{itemize}
        \item For each newly created array, check if an array of the same size already exists in \( L \).
        \item If a duplicate array size is found, merge the two arrays to form an array of the next higher power of 2.
        \item Repeat this merging process (cascading) until all resulting arrays have unique power-of-2 sizes and can be added back to \( L \) without duplicates.
        \item The cascading merges take \( O(n) \) time in the worst case because each element is involved in at most one merge per level. The total merge cost across all levels is given by:
        \[
        \sum_{i=0}^{\log n} O(2^i) = O(n)
        \]
    \end{itemize}

    \item \textbf{Insert Resulting Arrays into \( L \)}:
    \begin{itemize}
        \item Add the final set of unique power-of-2-sized arrays back into \( L \), ensuring that each level in \( L \) has only one array of each power-of-2 size.
    \end{itemize}
\end{enumerate}

\textbf{Overall Time Complexity}: The total time complexity for \texttt{Delete(x)} is \( O(n) \).
\newpage

\section*{Question 3 (written by Nena,  verified by Natalia)}
A brief description of the algorithm
\begin{itemize}
    \item We define G as an adjacency list representing the vertices and edges in the graph. The color can be white (not discovered), grey (discovered but not explored), and black (discovered and explored). The distance d is the distance from the nearest hospital. And the parent p is the node that discovered the current node.
    \item We will use the BFS algorithm to discover and explore all nodes to find the distance from the nearest hospital. \\
        \begin{algorithm} [H]
        \caption{BFS(G):}
        \begin{algorithmic}[1]
        \State Q $\gets empty$ 
        \For{each $v \in V$}
            \If{$v \in$ hospitals}
                \State color[v] $\gets grey$
                \State d[v] $\gets 0$
                \State p[v] $\gets NIL$
                \State ENQ(Q, h)
            \Else
                \State color[v] $\gets white$
                \State d[v] $\gets \infty$
                \State p[v] $\gets NIL$
            \EndIf
        \EndFor 
        
        \While{Q is not empty}
            \State u $\gets$ DEQ(Q)
            \For{each $(u, v) \in E$ - {H}}
                \If{color[v] == $white$}
                    \State color[v] $\gets grey$
                    \State d[v] $\gets d[u] + 1$
                    \State p[v] $\gets p[u]$
                    \State ENQ(Q, v)
                \EndIf
            \EndFor 
        \EndWhile
        \State color[u] $\gets black$
        \end{algorithmic}
        \end{algorithm}
     \item The reason why this algorithm is correct is because, high level, we are searching for houses that are distance 0, 1, 2... from hospitals in increasing order.
     \item \textbf{Predicate: }After processing all nodes at depth x, the queue will contain only nodes at depth x + 1 (distance of x + 1 from the nearest hospital).
    \item \textbf{Base Case: }Initially, we go through each vertex in the graph. For hospitals, we set their distance to 0, color them grey, and enqueue them, as they are the starting points for the BFS. For houses, we set the distance to infinity (unknown), parent to NIL, and color them white. The predicate holds at the start since the queue only contains hospitals, which are at distance 0.
    \item \textbf{Induction Step: } We assume that all the nodes in the queue have a d = x value that is the closest distance to a hospital [IH]. Note that the shortest path theorem still holds with multiple sources, being the multiple hospitals we enqueued in the beginning. If it is equally close to two hospitals, then it will depend on the order which the nodes are enqueued, but the min distance will not be affected. When we process these nodes, we explore their neighbors. For each white neighbor, we set its distance to x + 1, mark it grey, and enqueue it. Therefore, once we explore all nodes that have a d = x, we will only have nodes that have a d = x + 1 that are grey nodes, as previously grey and black nodes were not regarded. Therefore, the predicate holds. 
    \item We know that since to initialize each vertex to a color, distance, and parent takes constant time (this is the for loop) and there are $|V|$ vertices, O($\|V\|$). We also know that the while loop at most traverses through all the edges (with BFS it is possible to have unreachable nodes and edges), so this is O($\|E\|$). Therefore, BFS has a runtime of O($\|V\|$ + $\|E\|$).
\end{itemize}
\end{document}
